\documentclass{article}
\usepackage{amsmath, amssymb}
\title{\textbf{5.3 The Mean Value Theorem}}
\begin{document}
	\maketitle
	\textbf{Exercise 5.3.3.} Let $h$ be a differentiable function defined on the interval $[0, 3]$, and assume that $h(0) = 1$, $h(1) = 2$, and $h(3) = 2$.
	\begin{itemize}
		\textbf{a)} Argue that there exists a point $d \in [0, 3]$ where $h(d) = d$. \\
		\textbf{Proof:} Take the function $g(x) = h(x) - x$. Given that $g(1) = 1$ and $g(3) = -1$, then by the Intermediate Value Theorem, $g(d) = 0$ for some $d$ which implies that $h(d) = d$ as desired. \\ \\
		\textbf{b)} Argue that at some point $c$ we have $h'(c) = \frac{1}{3}$. \\
		\textbf{Proof:} By the Mean Value Theorem we have $h'(c) = \frac{h(3) - h(0)}{3 - 0} = \frac{2 - 1}{3} = \frac{1}{3}$ for some $c \in [0, 3]$.\\ \\
		\textbf{c)} Argue that $h'(x) = \frac{1}{4}$ at some point in the domain. \\
		\textbf{Proof:} By Rolle's Theorem we know that $g'(c) = 0$ for some $c \in [1, 3]$ and by \textbf{b)} we know that $g'(d) = \frac{1}{3}$ for some $d \in [0, 3]$. Now, by Darboux's Theorem on the interval $[0, 3]$, we can conclude that $h'(t) = \frac{1}{4}$ for some $t \in [0, 3]$.
	\end{itemize}
	\textbf{Exercise 5.3.5.} A fixed point of a function $f$ is a value $x$ where $f(x) = x$. Show that if $f$ is differentiable on an interval with $f'(x) \neq 1$, then $f$ can have at most one fixed point. \\ \\
	\textbf{Proof:} Assume that $f$ has two fixed points $(x_1, f(x_1))$ and $(x_2, f(x_2))$. By the Mean Value Theorem we have $f'(c) = \frac{f(x_1) - f(x_2)}{x_1 - x_2} = 1$ which leads to a contradiction. Thus, $f$ can have at most one fixed point. \\ \\
	\textbf{Exercise 5.3.7. a)} Recall that a function $f : (a, b) \to \mathbb{R}$ is increasing on $(a, b)$ if $f(x) \leq f(y)$ whenever $x < y$ in $(a, b)$. Assume $f$ is differentiable on $(a, b)$. Show that $f$ is increasing on $(a, b)$ if and only if $f'(x) \geq 0$ for all $x \in (a, b)$. \\ \\
	\textbf{Proof:} $\Longrightarrow$ Assume that $f$ is increasing. Now, suppose that for some $c$ in the domain we have $f'(c) < 0$. By the Mean Value Theorem we know that $f'(c) = \frac{f(e) - f(d)}{e - d}$ for some $a \leq d < e \leq b$. We now have $f(e) - f(d) < 0$, but this is a contradicition since that'd imply $f(e) < f(d)$. Thus, if $f(x)$ is increasing then $f'(x) \geq 0$. \\ 
	$\Longleftarrow$ Assume that $f'(x) \geq 0$ for all $x \in (a, b)$. Now suppose that $f(e) < f(d)$ for some $a \leq d < e \leq b$. By employing the technique shown in the previous direction, we can easily see that this would imply $f'(c) < 0$ for some $c \in (a, b)$, which is a contradiction. Thus, we can see that if $f'(x) \geq 0$ then $f(x) \leq f(y)$ for all $a \leq x < y \leq b$. This completes from the other direction and we can now conclude that $f$ is increasing on $(a, b)$ if and only if $f'(x) \geq 0$ for all $x \in (a, b)$. \\
	\begin{itemize}
		\textbf{b)} Show that the function
		$$g(x) = \begin{cases}
			\frac{x}{2} + x^2\sin(\frac{1}{x}) & \text{ if } x \neq 0 \\
			0 & \text{ if } x = 0
		\end{cases}$$
		is differentiable on $\mathbb{R}$ and satisfies $g'(0) > 0$. Now, prove that $g$ is not increasing over any open interval containing 0. \\
		\textbf{Proof:} By the definition of the derivative, we have $g'(0) = \lim_{x\to 0} \frac{g(x) - g(0)}{x - 0} = \lim_{x \to 0} \frac{g(x)}{x} = \lim_{x \to 0} \frac{1}{2} + x\sin(\frac{1}{x})$. Now, by the Algebraic Limit Theorem, we can see that $g'(0) = \frac{1}{2}$. \\
		For $x \neq 0$ we have $g'(x) = \frac{1}{2} - \cos(\frac{1}{x}) + 2x\sin(\frac{1}{x})$. Now, we need to find a sequence $(x_n)$ converging to 0 such that $g'(x_n) < 0$, the sequence $x_n = \frac{1}{2n\pi}$ satisfies this. Thus, there is no open interval around 0 where $g'(x) \geq 0$, and by the previous proof, $g'$ is not increasing on any intervak containing 0.
	\end{itemize}
\end{document}
