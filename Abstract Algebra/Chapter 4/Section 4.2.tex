\documentclass{article}
\usepackage{amsmath, amssymb}
\title{\textbf{4.2 Groups Acting on Themselves by Left Multiplication - Cayley's Theorem}}
\begin{document}
	\maketitle
	\textbf{Exercise 4.2.6.} Let $r$ and $s$ be the usual generators for the dihedral group of order 8 and let $N = \langle r^2 \rangle$. List the left cosets of $N$ in $D_8$ as $1N, rN, sN$ and $srN$. Label these cosets with the integers 1, 2, 3, 4 respectively. Exhibit the image of each element of $D_8$ by left multiplication on the set of 4 left cosets of $N$ in $D_8$. Deduce that this representation is not faithful and prove that $\pi_N(D_8)$ is isomorphic to the Klein 4-group. \\
	The images are: 
	\begin{itemize}
		$\sigma_r(1) = 2 \\
		\sigma_r(2) = r^2N = N = 1 \\
		\sigma_r(3) = rsN = sr^{-1}N = sr^3N = sr(r^2N) = srN = 4 \\
		\sigma_r(4) = rsrN = sN = 3 \\
		\sigma_s(1) = sN = 3 \\
		\sigma_s(2) = srN = 4 \\
		\sigma_s(3) = ssN = N  = 1 \\
		\sigma_s(4) = ssrN = rN = 2$
	\end{itemize}
	Because $\sigma$ is a homomorphism and we know the images of the generators of $D_8$ ($r$ and $s$) we can deduce the image of every other element in $D_8$ through the multiplication of images (e.g., $\sigma_{rs} = \sigma_r\sigma_s$). \\
	It is easy to see that $r^2$ is in the kernel of the action and thus we deduce that the representation is not faithful. \\
	As we've shown, the subgroup of $S_{D_8}$ that $\pi_N$ maps to can be generated by $\sigma_r = (1\; 2)(3\; 4)$ and $\sigma_s = (1\; 3)(2\; 4)$ which is the exact same subgroup that we've seen the Klein 4-group isomorphic to, thus $\pi_N \cong K_4$ where $K_4$ is the Klein 4-group.
\end{document}
